\chapter{Literature Review and Methodology}
\label{chap:review}

\section{State of photonics}
\section{Properties of III-V materials}
\section{\texorpdfstring{III-V-on-\acs{si} integration routes}{III-V-on-Si integration routes}}
The various integration routes of III-V semiconductor on Si can be at first divided in 2 broad categories: heterogeneous and homogeneous integration.
\subsection{Heterogeneous integration}
Heterogeneous integration refers to the indirect integration of III-V material grown on a different, lattice matched, substrate on a Si wafer. 
\par
The main advantage of these techniques is the elimination of lattice mismatch as a source of defects during epitaxial growth. Furthermore it allows a selection of the best performing structures to be transferred onto the Si substrate3,4, and does not have to compromise on the growth parameters or material systems for the sake of CMOS compatibility (except for bonding temperature). The various methods that fall under this definition form, all together, a mature technology that already finds application in industry4, therefore benefitting from a few years of industrial optimization.
\par
On the other hand, this type of integration requires the growth of III-V to take place in a different fabrication line that has to maintain the high precision and cleanliness standard of the main CMOS silicon line, resulting in a rather large capital investment. Furthermore, most classic transfer steps can result in a material with a more irregular geometry (wafer bow, surface roughness after etching), or presenting transfer-related defects5, or require an extra bonding layer5,6, and do not allow for nm precise integration4,7. The most advanced techniques that circumvent most of these quality issues are, however, too slow to provide a competitive transfer time for large, densely integrated, 8’’ production wafers4,7.
\subsection{Homogeneous Integration}
Homogeneous integration refers to the direct integration by epitaxial growth of III-V semiconductors on Si. 
It naturally provides advantages such as the extremely high spatial precision and accuracy for the integration of small devices over wafer scale substrates, which can be achieved in a shorter time frame when compared to its heterogenous analogues4. It has the potential of being a more economical alternative to heterogeneous integration: if the growth process can be integrated with current CMOS processes, its implementation in a CMOS line would render having an entire dedicated III-V fab running in parallel with a Si based fab unnecessary, especially if the use of III-V electronics is envisioned at the same time as III-V photonics4,therefore potentially reducing the capital cost required to implement this efficient light emitting and absorbing material in different devices6.
The main disadvantage is related to the high lattice mismatch between Si and most III-V materials, meaning strain-relaxing defects are nucleated at the heterointerface2: these defects have a very detrimental effect on the performance and lifetime of devices8,9, and act as scattering or recombination centres10. The polar nature of III-V atomic bonds when compared to non-polar Si-Si bonds can create anti-phase boundaries during growth on certain Si facets2, and most known surface treatments to eliminate the nucleation sites for these defects occur in temperature ranges that are incompatible with the CMOS process11. Furthermore, materials that are known to grow at a higher temperature, such as III – nitrides, and others that might have detrimental effects on passivating SiO2 layers, such as gallium11, also pose compatibility problems with the CMOS process. Another key obstacle, especially relating to the direct growth of micro and nanostructures, is the stricter requirements for reproducibility and reliability of the process.
\subsection{Comparison between homogeneous integration routes}
What follows is a table comparing the advantages and disadvantages of various homogenous integration techniques. The table predominantly focuses on what can be accomplished in the growth reactor, and briefly mentions techniques that require simple or complex substrate preparation.


\begin{sidewaystable}
    \centering
\begin{longtable}{p{0.1\textwidth}|p{0.42\textwidth}|p{0.42\textwidth}}
    Type & Advantages & Disadvantages \\ \hline \hline
    Planar growth & 
    \begin{itemize}
        \item Extremely simple substrate preparation steps.
        \item Allows Stranski-Krastanov growth mode for quantum dot formation12,13
    \end{itemize} & 
    \begin{itemize}
        \item Often leads to wafer bow or warp due to thermal expansion coefficient mismatch14,15
        \item Can require 1+ micrometre thick strain management layer before device growth (depending on mismatch)15,16
        \item Widespread defects are common in this kind of growth17
    \end{itemize} \\ \hline
    Selective Area Growth (SAG) & 
    \begin{itemize}
        \item Substrate preparation ranges from relatively simple to somewhat complex
        \item Allows the growth of nanostructures16
        \item Allows the growth of core-shell wires18
        \item Allows position control18
        \item No wafer warp or bow as a result of the III-V material
        \item Allows the growth of various device structures, but only vertically19,20
    \end{itemize}  & 
    \begin{itemize}
        \item Sidewall growth can be minimized but not  eliminated, resulting in undesired heterointerfaces and possible loss of composition control in complex systems.
        \item Growth mainly happens in the vertical direction
    \end{itemize}  \\ \hline
    VLS/Droplet epitaxy (vertical nanowires) & 
    \begin{itemize}
        \item Simple substrate preparation process
        \item Allows for nanostructure growth21
        \item No wafer warp or bow as a result of the  III-V material
        \item Allows for sophisticated growth studies such as in-situ TEM imaging22–24, leading to the possibility to refine growth recipes to the point where phase control can be achieved25–27.
        \item Can be tuned to achieve high directionality of the growth and possibility to incorporate heterostructures with minimal or no undesired side growth24,27
    \end{itemize}  & \begin{itemize}
        \item Limited position control27
        \item Nanostructure geometry limited to nanowires21
        \item Reservoir effect complicates composition control in ternary and quaternary compounds28
    \end{itemize} \\ \hline
    \caption{Caption}
    \label{tab:methods1}
\end{longtable}
\end{sidewaystable}
    \begin{sidewaystable}
    \centering
    \begin{longtable}{p{0.1\textwidth}|p{0.42\textwidth}|p{0.42\textwidth}}
    Type & Advantages & Disadvantages \\ \hline \hline
    Selective Area Growth with Aspect Ratio Trapping & 
    \begin{itemize}
        \item Allows position control for the III-V growth29
        \item Highly reduced bow and warp effect on the wafer after deposition
        \item Aspect ratio trapping (ART) allows a base level of defect control29
    \end{itemize}  & \begin{itemize}
        \item Defects propagating in any direction that shares the plane of the side walls are not filtered2
        \item Multiple nucleation points can result in grain boundaries, and their elimination requires extra process steps30
    \end{itemize} \\ \hline
    Template Assisted Selective Epitaxy (TASE) & 
    \begin{itemize}
        \item Allows the growth of a very large variety of both vertical and horizontal nanostructures while maintaining extremely accurate geometry and  position control31–33
        \item Superior defect control, even compared with ART: far from the growth interface defects can be effectively limited to twin planes34, which in some cases can too be eliminated35
        \item Potential for excellent facet and composition36,37 control in nanowires, with no side growth when growing heterostructures38 
        \item No wafer warp or bow as a result of the III-V material
        \item Allows for phase control in determinate growth geometries39
    \end{itemize} & 
    \begin{itemize}
        \item Substrate preparation takes place in a complex multi-step process
        \item Can suffer from parasitic growth
        \item Some height limitations in terms of core shell structures, which can’t be grown horizontally, and are de-facto limited to microdisk form factors32
    \end{itemize} \\ \hline
    \caption{Caption}
    \label{tab:methods2}
\end{longtable}
\end{sidewaystable}

It should be noted that often, in papers from Dr. Lau’s group, the growth method they use is called lateral aspect ratio trapping (LART)34,40: this is in essence a hybrid between SAG-ART and TASE, and can be described as a type of TASE with particularly wide (sometimes several micrometres) templates, therefore I have grouped it with TASE in the above table. It should be noted, however, that while LART offers most of the same advantages that TASE offers, it is more susceptible to the formation of grain boundaries due to multiple nucleation (which does not take place in TASE) and has reduced defect trapping in the wide direction. On the other hand, due to their etch-based approach, they can achieve more intricate quantum well positioning in their microdisk lasers than what is possible with our TASE approach for microdisk growth.

\section{\texorpdfstring{State of \acl{tase}}{State of template assisted selective epitaxy}}
\section{Characterization of nano- and microstructures}
Talk about defect types (what is an artp is very important) and therefore about the lattices

\section{Article 2}

Due to the presence of the SiO2 template, the quantum wells
develop only axially and not laterally. This results in improved
composition control in heterolayers of ternary III−V
compounds incorporated in the nanowires\cite{Borg2019} and therefore is
expected to allow for better control of the emission spectra.