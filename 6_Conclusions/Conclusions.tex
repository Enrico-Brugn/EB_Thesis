\chapter{Conclusions}
\label{chap:conclusions}

The work presented in this thesis focused on the identification of the conditions under which the \acf{mocvd} of III-V semiconductors in \acf{tase} templates results in a growth front consisting of a single \hkl{1 1 1}\(_B\) facet, the integration of thin heterostructures capable of quantum confinement in a \acs{tase} nanowire, the properties of the nanowires grown with this method, and its yield. The study initially only focussed on the lattice matched \ce{In0_.55Ga0_.45As} and \acs{inp} semiconductors.

The process began by analysing a single heterostructure consisting of two material segments, which showed the difference in growth front morphology between \acf{ingaas} grown with a low V / III ratio and \acf{inp} grown with a high V / III ratio. As seen in Figure~\ref{subfig:sample1_cross-sec}, the low V / III \acs{ingaas} layer grew with the arrowhead growth front, well known in literature \cite{Knoedler2017}, consisting of two \hkl{1 1 0} faces on the upper half of the wire and a \hkl{1 1 1} facet at the bottom. Conversely, in the high V / III ratio \acs{inp} segment the growth rate was increased along the \hkl{1 1 0} directions, resulting in the annihilation of the two top \hkl{1 1 0} facets to the benefit of the bottom \hkl{1 1 1} facet. Therefore, it became clear that a high V / III ratio \acs{inp} growth regimen achieves the goal of stabilising the single-facet growth front.

However, when quick switching between precursors was attempted to create a thin heterostructure, the resulting nanowire, seen in Figure~\ref{fig:sample2_stem}, showed a high composition of elements of the III group intermixing in the heterointerfaces coupled with the integration delay between elements of the III and V group. This caused the III and V concentration gradients to be misaligned. However, the V group element \acf{eds} map also showed potential for well-defined quantum confinement structures.

Some recipe adjustments were made to build on the stabilisation of the \hkl{1 1 1} facet shown to occur in \acs{inp} growth with high V / III ratios, solve III - V integration delay, and improve III group element definition at the heterointerface. The introduction of hold steps in the recipe was shown to achieve all of these objectives in Figure~\ref{fig:s3_ov}. Later experiments showed that increasing the length of the \acs{inp} stabilisation segment and reducing the deposition time of the \acs{ingaas} nucleation layer reduced variability in the stabilisation results. This created a platform allowing for consistent III group element integration across the entire growth front and for constant growth rates, simplifying layer thickness control in quantum confinement structures.

Which of the two available \hkl{1 1 1} facets would stabilise as a growth front in the \hkl(0 0 1) \acf{soi} wafer remained, however, a stochastic problem. A switch to a \hkl(1 1 0) \acs{soi} wafer was made to allow for a consistent selection of best aligned \hkl{1 1 1} in-plane facet. For comparison, the device layer \acs{si} of the \hkl(0 0 1) wafer was \qty{220}{\nano\metre} thick, while the new \hkl(1 1 0) wafer had a device layer thickness of \qty{70}{\nano\metre}. As the device layer's thickness determines the nanostructures' height, the new nanowires' growth rates changed slightly compared to those of the old nanowires but remained comparable.

Growth in a competitive environment was studied on an area of the chip with diffused parasitic nucleation. Here, the nanowires competed with each other, but also with crystals unconstricted by the \acs{tase} template. The presence of these crystals, which acted as larger and larger capture centers for precursors as their surface area grew, changed the effective V / III ratio inside the \acs{tase} template. As a result, the growth rates of each consecutive material layer started to diminish despite the growth front progressing towards the template opening - a situation which was accompanied by a growth rate increase in previous nanowires. Finally, even the stabilising effect given by the high V / III ratio broke down and a multi-faceted growth front was re-established. This effect should be accounted for in the growth of dense sets of \acs{tase} structures with different shapes.

The \hkl{1 1 1}\(_B\) stabilisation method proved to be an exciting avenue to grow defect-free single crystals from simultaneous nucleation on multiple seeds, as shown in Figure~\ref{fig:merge_ov}. The defect-free merging of the three seed crystals was achieved thanks to the layer-by-layer growth on \hkl{1 1 1} atomic planes enabled by the highly controlled growth regimen established during the deposition of \acs{inp} with a high V / III ratio.

The deposition of \acs{ingaas} layers with a thickness of four atomic bi-layers was an achievement at the very limit of what can be achieved with \acf{mocvd}, highlighting both the merits and limitations of the recipe developed in this thesis. It also showed how any heterointerface achievable with this method is finite with a width larger than \qty{2}{\nano\metre}. The quantum well consisting of lattice-mismatched \acf{inas} between two \acs{inp} segments was analysed with \acf{gpa}. This thin layer showed a very small change in lattice constant, synonymous with local strain.

A survey was carried out across two of the samples to establish yield statistics for \acs{tase} on \hkl(1 1 0) \acs{soi} using the single-facet \hkl{1 1 1} growth front method. \num{240} nanowire arrays were imaged, resulting in a dataset of \num{15840} nucleation sites. Yield calculations showed how \num{14660} wires grew with the intended end facet, indicative of an early stabilisation of the \hkl{1 1 1} facet as the growth front. The three most common reasons for defective wires were:

\begin{enumerate}
    \item nucleation sites hidden by parasitic crystals: \num{487} sites, \qty{41.27}{\percent} of defective and \qty{3.07}{\percent} of total sites;
    \item nucleation sites ending with the wrong facet configuration: \num{368} sites, \qty{31.19}{\percent} of defective and \qty{2.32}{\percent} of total sites;
    \item significantly shot wires: \num{211} sites, \qty{17.88}{\percent} of defective and \qty{1.33}{\percent} of total sites.
\end{enumerate}



\section{Future Directions}

can address twinning (already have ref of twin free growth on 111)

can select 110 by using a method similar to goswami et al (first go up and then laterally) with a si (110) soi

further fib-sem study on etch-back and relation of seed shape to growth front morfology (among other things, can you eliminate the 2 possibilities of the (1 1 1) facet on (0 0 1) soi by changing the seed shape?)

thickness limit (and as a function of how much the facet is stabilised)


\section{old conclusions}

\subsection{properties}



The most significant impact on growth rates was observed in samples grown in a competitive environment. The presence of many parasitic crystals deeply affected the growth of \acs{tase} nanowires and the growth regimen within the template. The effective V / III ratio alteration in this environment was enough to cause the wire to lose its single-faceted growth front. This effect should be considered when attempting dense integration of \acs{tase} structure, especially if they differ in shape and size.

Further yield calculations for this multi-seed single structure and a study of their performance as a function of size are interesting research avenues. 

Further optimisation of the growth recipe for the deposition of \acs{gaas} and \acs{gasb} in thin layers will complement these findings and expand the knowledge base past the lattice-matched \acs{inp}-\acs{ingaas} case.

Further investigation to find a \textit{p} dopant that also does not interfere with the facet-stabilisation effect would complete the doping toolset for \textit{p} - \textit{i} - \textit{n} structures.

\subsection{yield}

Nucleation-related issues represented two-thirds (\num{796} or \qty{67.46}{\%}) of the recorded defects, mostly due to surface treatment-related issues, such as loss of selectivity and seed surface conditions. This points to the strength of the facet stabilisation method as a reliable and controlled approach to the growth of monolithically integrated III-V crystals on chip-sized surfaces, supplementing observations on a more local level regarding the merging of multiple crystals in a single structure shown in Section~\ref{sec:merge}.

The survey of the two samples that resulted in the statistical data on growth yield also produced a labelled dataset containing \num{240} \acs{sem_m} images \cite{dataset}, publicly available for use under the CC BY 4.0 licence \cite{CCBY40}. In this chapter, the dataset was used to train a classifier that could distinguish between the two orientations of the nanowire arrays. The classifier achieved good performance in discriminating perfect wires between tilted and parallel but struggled with the identification of defective wires. This can be attributed to the high imbalance in the dataset, which can be resolved by collecting more images of defective wires of both kinds to supplement the currently available data. Further improvements can also be made to the wire-splitting algorithm, where a different approach might yield a more reliable solution to the determination of the cut points.