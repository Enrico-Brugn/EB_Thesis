\chapter{Conclusions}
\label{chap:conclusions}

The work presented in this thesis focused on the identification of the conditions under which the \acf{mocvd} of III-V semiconductors in \acf{tase} templates results in a growth front consisting of a single \hkl{1 1 1}\(_B\) facet, the integration of thin heterostructures capable of quantum confinement in a \acs{tase} nanowire, the properties of the nanowires grown with this method, and its yield. The study initially only focussed on the lattice matched \ce{In0_.55Ga0_.45As} and \acs{inp} semiconductors.

The process began by analysing a single heterostructure consisting of two material segments, which showed the difference in growth front morphology between \acf{ingaas} grown with a low V / III ratio and \acf{inp} grown with a high V / III ratio. As seen in Figure~\ref{subfig:sample1_cross-sec}, the low V / III \acs{ingaas} layer grew with the arrowhead growth front, well known in literature \cite{Knoedler2017}, consisting of two \hkl{1 1 0} faces on the upper half of the wire and a \hkl{1 1 1} facet at the bottom. Conversely, in the high V / III ratio \acs{inp} segment the growth rate was increased along the \hkl{1 1 0} directions, resulting in the annihilation of the two top \hkl{1 1 0} facets to the benefit of the bottom \hkl{1 1 1} facet. Therefore, it became clear that a high V / III ratio \acs{inp} growth regimen achieves the goal of stabilising the single-facet growth front.

However, when quick switching between precursors was attempted to create a thin heterostructure, the resulting nanowire, seen in Figure~\ref{fig:sample2_stem}, showed a high composition of elements of the III group intermixing in the heterointerfaces coupled with the integration delay between elements of the III and V group. This caused the III and V concentration gradients to be misaligned. However, the V group element \acf{eds} map also showed potential for well-defined quantum confinement structures.

Some recipe adjustments were made to build on the stabilisation of the \hkl{1 1 1} facet shown to occur in \acs{inp} growth with high V / III ratios, solve III - V integration delay, and improve III group element definition at the heterointerface. The introduction of hold steps in the recipe was shown to achieve all of these objectives in Figure~\ref{fig:s3_ov}. Later experiments showed that increasing the length of the \acs{inp} stabilisation segment and reducing the deposition time of the \acs{ingaas} nucleation layer reduced variability in the stabilisation results. This created a platform allowing for consistent III group element integration across the entire growth front and for constant growth rates, simplifying layer thickness control in quantum confinement structures.

Which of the two available \hkl{1 1 1} facets would stabilise as a growth front in the \hkl(0 0 1) \acf{soi} wafer remained, however, a stochastic problem. A switch to a \hkl(1 1 0) \acs{soi} wafer was made to allow for a consistent selection of best aligned \hkl{1 1 1} in-plane facet. For comparison, the device layer \acs{si} of the \hkl(0 0 1) wafer was \qty{220}{\nano\metre} thick, while the new \hkl(1 1 0) wafer had a device layer thickness of \qty{70}{\nano\metre}. As the device layer's thickness determines the nanostructures' height, the new nanowires' growth rates changed slightly compared to those of the nanowires grown on the \hkl{0 0 1} substrate but remained comparable.

Growth in a competitive environment was studied on an area of the chip with diffused parasitic nucleation. Here, the nanowires competed with each other, but also with crystals unconstricted by the \acs{tase} template. The presence of these crystals, which acted as larger and larger capture centers for precursors as their surface area grew, changed the effective V / III ratio inside the \acs{tase} template. As a result, the growth rates of each consecutive material layer started to diminish despite the growth front progressing towards the template opening - a situation which was accompanied by a growth rate increase in previous nanowires. Finally, even the stabilising effect given by the high V / III ratio broke down and a multi-faceted growth front was re-established. This effect should be accounted for in the growth of dense sets of \acs{tase} structures with different shapes.

The \hkl{1 1 1}\(_B\) stabilisation method proved to be an exciting avenue to grow defect-free single crystals from simultaneous nucleation on multiple seeds, as shown in Figure~\ref{fig:merge_ov}. The defect-free merging of the three seed crystals was achieved thanks to the layer-by-layer growth on \hkl{1 1 1} atomic planes enabled by the highly controlled growth regimen established during the deposition of \acs{inp} with a high V / III ratio.

The deposition of \acs{ingaas} layers with a thickness of four atomic bi-layers was an achievement at the very limit of what can be accomplished with \acf{mocvd}, highlighting both the merits and limitations of the recipe developed in this thesis. It also showed how any heterointerface achievable with this method is finite with a width larger than \qty{2}{\nano\metre}. The quantum well consisting of lattice-mismatched \acf{inas} between two \acs{inp} segments was analysed with \acf{gpa}. This thin layer showed a very small change in lattice constant, synonymous with local strain.

A survey was carried out across two of the samples to establish yield statistics for \acs{tase} on \hkl(1 1 0) \acs{soi} using the single-facet \hkl{1 1 1} growth front method. \num{240} nanowire arrays were imaged, resulting in a dataset of \num{15840} nucleation sites. Yield calculations showed how \num{14660} wires grew with the intended end facet, indicative of an early stabilisation of the \hkl{1 1 1} facet as the growth front, and resulting in a total yield of \qty{92.55}{\%}. The three most common reasons for defective wires were:

\begin{enumerate}
    \item nucleation sites hidden by parasitic crystals: \num{487} sites, \qty{41.27}{\percent} of defective and \qty{3.07}{\percent} of total sites;
    \item nucleation sites ending with the wrong facet configuration: \num{368} sites, \qty{31.19}{\percent} of defective and \qty{2.32}{\percent} of total sites;
    \item significantly shot wires: \num{211} sites, \qty{17.88}{\percent} of defective and \qty{1.33}{\percent} of total sites.
\end{enumerate}

Together, nucleation-related issues represented \qty{67.47}{\percent} of the recorded defects, cementing the facet stabilisation method as a reliable approach to the growth of III-V semiconductor on \acl{si}, since further optimisation of pre-\acs{mocvd} surface treatments would eliminate these defects.

The data set used in this yield study is publicly available \cite{dataset} under the terms of the CC BY 4.0 licence \cite{CCBY40}. In it, each wire and parasitic crystal has been given one of five labels: "Parasitic", "Wire\_Straigh\_Perfect", "Wire\_Straigh\_Defect", "Wire\_Tilted\_Perfect", and "Wire\_Tilted\_Defect". In this thesis, the images that make up the data set were first split down to individual wires using an automated splitting algorithm \cite{code} and then used in a simple classifier. Training metrics (see the confusion matrix in Figure~\ref{subfig:confusion_matrix}) showed a marked classification ability for parasitic crystals and perfect wires, whether straight or tilted. However, due to data set imbalance, the model's ability to classify defective wires was not very high.

\section{Future directions}

In some areas, the addition of more data can complement the research presented in this thesis. Statistical data on the growth yield of multi-seed structures, recipe optimisation for the deposition of \acf{gaas} and \acf{gasb} layer, and investigation of the effect of \textit{p} dopants on growth would enrich this work. Further development on the wire splitting algorithm presented in Chapter~\ref{chap:yield_analysis} and perhaps a different approach could also improve its performance. 

Studies on the controlled merging of crystals grown with the method presented in this thesis could explore the effect of larger platelet areas and seed spacing on the final defect concentrations. Constructing an appropriately sized cell for in-situ \acf{tem_m} analysis could empower this approach by allowing real-time imaging of the growth dynamics occurring at the merging stage of the three crystals.

Politypism caused by the high density of a single type of defect, the \acf{rtp}, remained an issue in all samples and is a by-product of the \hkl{1 1 1} stabilisation method. Two research avenues can be explored to reduce the amount of \acs{rtp}s in \acs{tase} nanowires or eliminate them. The first is to carefully adapt the growth recipe by tuning the V / III ratios and growth temperatures depending on the growth stage, as seen in both \acf{sag} \cite{Chi2013} and \acf{vls} epitaxy \cite{Joyce2007}. In-situ \acs{tem_m} could also be instrumental in shedding light on the dynamic of twin-plane formation in enclosed templates. The second avenue would be to apply the template design used by Goswami et al. in \cite{Goswami2020} on the \hkl(1 1 0) \acs{soi} and focus on stabilising a single \hkl{1 1 0} facet as the growth front.

Another avenue of research could explore the effect of the morphology of the \acl{si} seed on the morphology of the growth front. A method for selecting a specific \hkl{1 1 1} facet between the two available during growth along a \hkl<1 1 0> direction on a \hkl(0 0 1) \acs{soi} could see better growth reproducibility on this commonly used substrate. Furthermore, the use of a different etchant to create a \hkl{1 1 0} seed facet coupled with growth conditions enhancing the growth in \hkl{1 1 1} directions could also achieve a single-facet heterointerface in \acs{tase} nanowire growth, and a yield study for comparison with the \hkl{1 1 1} stabilisation method used in this work would enrich the knowledge on the subject.

Further research on the stability of the single-facet growth front in \acs{tase} templates could shed light on any potential mechanisms with which a return to a multi-faceted growth front could occur, such as critical layer thicknesses of lattice-mismatched semiconductors.

Another important avenue of characterisation which was not explored in the present work is the analysis of the electrical and electro-optical properties of the nanowires containing quantum wells fabricated with the facet stabilisation method. For example, their introduction in the \textit{i} region of a \textit{p} - \textit{i} - \textit{n} photodetector is expected to lead to reduced dark currents \cite{Xue2021}, and measurements to confirm these findings would be a natural next step in the integration of this growth regimen in the fabrication of \acs{tase}-based \acs{pic}s.