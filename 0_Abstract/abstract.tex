\chapter{Abstract}

Integration of \acl{pic}s (\acs{pic}s) serving as interconnects on the \acl{si} electronic platform is gaining momentum as computational requirements increase. Due to material compatibility, \acl{si}-based photonic components remain the main choice for industrial integration. However, research focused on integrating III-V semiconductor-based photonic components has made great progress in tackling the problem of lattice mismatch, which has hindered the use of this material on \acl{si} platforms so far.

A promising monolithic method for the integration of III-V components in \acs{cmos} wafers is \acf{tase}, which employs a \acf{sio2} template to guide the \acf{mocvd} of III-V devices from a \acl{si} seed defined in the device layer of a \acf{soi} wafer. \acs{tase} enables high defect control but is still affected by a multifaceted growth front, which limits its compositional control and efficient integration of quantum confinement structures.

This work focuses on the monolithic growth of III-V nanowires from \acl{si}, and, in particular, on the control of the morphology of the growth front and the sharpness of heterointerfaces as enablers for the creation of superlattices embedded in the nanowires. Stabilisation of a growth front consisting of a single \hkl{1 1 1}\(_B\) facet was achieved on two \acs{soi} wafers with \hkl(0 0 1) and \hkl(1 1 0) device layer crystallographic orientations by exploiting the growth of \acs{inp} with high precursor V / III ratios. With an overall growth yield of \qty{92.55}{\%} calculated on a sample of \num{15840} growth sites on two \hkl(1 1 0) chips, the method has proven to be very reliable even under laboratory conditions.

This method enabled the fabrication of \acs{ingaas} and \acs{inas} \acl{qw}s in a \acs{inp} matrix. Compositionally sharp heterointerfaces and perfect alignment of concentration profiles of the V and III group elements in the sub-\qty{10}{\nano\metre} layers were achieved by employing hold steps in the \acs{mocvd} recipe. The growth regime was estimated to be "layer-by-layer", with a very early stabilisation of a single \hkl{1 1 1}\(_B\) facet as the growth front, as single crystals resulting from the merging of crystals grown from three nucleation sites were observed with the methodology used in this study. These merged crystals had a crystalline quality fully comparable to that of crystals grown from a single seed.

Growth rate homogeneity and highly predictable heterointerface positioning achieved in this work are expected to aid in the monolithic integration of photonic devices on the wafer scale. Integration of quantum well structures demonstrated in this thesis into the intrinsic region \textit{p} - \textit{i} - \textit{n} photodiodes and in-depth electrooptical characterisation are the natural next step to improve the performance of \acs{tase}-based photodetectors. In conjunction with recent progress on \acs{tase} based modulators and micro disk lasers, this will eneable a fully \acs{tase}-fabricated \acs{pic}.