\chapter{Abstract}

Integration of \acl{pic}s serving as interconnects on the \acl{si} electronic platform is gaining momentum as computational requirements increase. Due to material compatibility, \acl{si}-based photonic components remain the main choice for industrial integration. However, research focused on integrating III-V semiconductor-based photonic components has made great progress in tackling the problem of lattice mismatch, which has hindered the use of this material on \acl{si} platforms so far.

A promising monolithic method for the integration of III-V components in \acs{cmos} wafers is \acf{tase}, which employs a \acf{sio2} template to guide the \acf{mocvd} of III-V devices from a \acl{si} seed defined in the device layer of a \acf{soi} wafer. \acs{tase} enables high defect control but is still affected by a multifaceted growth front, which limits its compositional control and efficient integration of quantum confinement structures.

This work focuses on the monolithic growth of III-V nanowires from \acl{si}, and, in particular, on the control of the morphology of the growth front and the sharpness of heterointerfaces as enablers for the creation of superlattices embedded in the nanowires.