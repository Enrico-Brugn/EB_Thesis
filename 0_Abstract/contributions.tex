\chapter{Contributions to the Thesis}

In the following flowcharts, the tasks in black are those I performed. Tasks that other staff and researchers kindly carried out on my request and the key external input for the pooling method in the image preprocessing are highlighted in red. The black rectangles represent multiple process steps, further explained in Section~\ref{chap:tase} for the initial nanofabrication process and in the experimental chapters for growth, characterisation, and data analysis. Rhomboids represent single process steps. The rounded boxes represent the flowcharts' start and end(s).

\begin{figure}[H]
\centering
\tikzsetnextfilename{nanofab_flow}
\begin{tikzpicture}
    \node[draw,
          blue,
	   rounded rectangle, 
	   minimum width=2.5cm,] (soi){Pristine SOI wafer (commercially available)};
    \node[draw,
          red,
          below = of soi,
          trapezium,
          trapezium left angle = 120,
          trapezium right angle = 120,
          trapezium stretches] (dicing) {Dicing};
    \node[draw,
          below = of dicing,
          trapezium,
          trapezium left angle = 120,
          trapezium right angle = 120,
          trapezium stretches] (mask) {Design of the lithography mask};
    \node[draw,
          red,
          right = of mask,
          trapezium,
          trapezium left angle = 120,
          trapezium right angle = 120,
          trapezium stretches] (optimask) {Fabrication of the optical mask};
    \node[draw,
          below = of mask,
	   minimum width=2.5cm,] (markers){Fabrication of e-beam markers};
    \node[draw,
          red,
          right = of markers,
          align = center,
          text width = 6cm] (exposure1) {\acs{w} sputtering | e-beam exposure};
    \node[draw,
          below = of markers,
	   minimum width=2.5cm,] (nanostructures){Definition of nanostructures};
    \node[draw,
          red,
          right = of nanostructures,
          trapezium,
          trapezium left angle = 120,
          trapezium right angle = 120,
          trapezium stretches] (exposure2) {e-beam exposure};
    \node[draw,
          below = of nanostructures,
	   minimum width=2.5cm,] (template){Etching and template oxide deposition};
    \node[draw,
          below = of template,
	   minimum width=2.5cm,] (openings){Creation of the openings in the template oxide};
    \node[draw,
          red,
          right = of openings,
          trapezium,
          trapezium left angle = 120,
          trapezium right angle = 120,
          trapezium stretches] (exposure3) {e-beam exposure};
    \node[draw,
          below = of openings,
	   minimum width=2.5cm,] (etch){Etch back of the sacrificial \acs{si} in the template};
    \node[draw,
          below = of etch,
	   minimum width=2.5cm,] (growth){Growth of III-V material inside the templates};
    \node[draw,
          right = of growth,
          rounded rectangle,
	   minimum width=2.5cm,] (end){III-V \acs{tase} nanostructures};

    \draw [-stealth](soi) edge (dicing);
    \draw [-stealth](dicing) edge (mask);
    \draw [-stealth, red] (mask) -- (optimask);
    \draw [-stealth](mask) edge (markers);
    \draw [stealth-, red] (markers) -- (exposure1);
    \draw [-stealth](markers) edge (nanostructures);
    \draw [stealth-, red] (nanostructures) -- (exposure2);
    \draw [-stealth] (nanostructures) edge (template);
    \draw [-stealth] (template) edge (openings);
    \draw [stealth-, red] (openings) -- (exposure3);
    \draw [-stealth] (openings) edge (etch);
    \draw [-stealth] (etch) edge (growth);
    \draw [-stealth] (growth) edge (end);
\end{tikzpicture}
\caption{Nanofabrication flowchart.}
\label{fig:nanofab_flow}
\end{figure}

\begin{figure}
    \centering
    \tikzsetnextfilename{characterisation_flow}
    \begin{tikzpicture}
        \node [minimum width=2.5cm](start){};
        \node[draw,
              left = of start,
	       rounded rectangle, 
	       minimum width=2.5cm,] (nanostructures){III-V \acs{tase} nanostructures};
        \node[draw,
	       below = of start, 
	       minimum width=2.5cm,] (fib){FIB lamella preparation};
        \node[draw,
	       below = of fib, 
	       minimum width=2.5cm,] (stem){\acs{stem_m} analysis};
        \node[draw,
	       rounded rectangle, 
	       below = of stem, 
	       minimum width=2.5cm,] (structural){Structural information};
        \node[draw,
	       rounded rectangle, 
	       right = of stem, 
	       minimum width=2.5cm,] (composition){Compositional information};
        \node[draw,
              red,
	       left = of structural, 
	       minimum width=2.5cm,] (extra_stem){\acs{tem_m} imaging | \acs{gpa}};
        \node[draw,
              red,
	       rounded rectangle, 
	       right = of start, 
	       minimum width=2.5cm,] (pl){Photoluminescence spectra};
        \node[draw,
	       above = of start, 
	       minimum width=2.5cm,] (sem){\acs{sem_m} imaging};
        \node[draw,
	       rounded rectangle, 
	       right = of sem, 
	       minimum width=2.5cm,] (yield){Growth yields};
        \node[draw,
	       above = of sem, 
              trapezium,
              trapezium left angle = 120,
              trapezium right angle = 120, 
	         text width=3.5cm,
              align = center,
              trapezium stretches] (labelling){Labelling of \acs{sem_m} images};
        \node[draw,
	       rounded rectangle, 
	       left = of labelling,
              align = center,
	         text width=4cm,] (dataset){Nanowire array \acs{sem_m} images dataset};
        \node[draw, 
	       above = of labelling, 
	       minimum width=2.5cm,] (algo){Image preprocessing algorithm};
        \node[draw,
              red,
	       right = of algo, 
              trapezium,
              trapezium left angle = 120,
              trapezium right angle = 120,
              trapezium stretches] (pool){Pooling method};
        \node[draw, 
	       rounded rectangle,
	       above = of algo, 
	       minimum width=2.5cm,] (classifier){Classifier for yield calculation};

        \draw [-stealth](nanostructures) edge (pl);
        \draw [stealth-stealth](sem) edge (fib);
        \draw [-stealth](fib) edge (stem);
        \draw [-stealth](fib) -| (extra_stem);
        \draw [red, -stealth](extra_stem) edge (structural);
        \draw [-stealth](stem) edge (structural);
        \draw [-stealth](stem) edge (composition);
        \draw [-stealth](sem) edge (yield);
        \draw [-stealth](sem) edge (labelling);
        \draw [-stealth](labelling) edge (algo);
        \draw [-stealth](labelling) edge (dataset);
        \draw [-stealth](algo) edge (classifier);
        \draw [red, stealth-](algo) edge (pool);
    \end{tikzpicture}
    \caption{Characterisation flowchart.}
    \label{fig:characterisation_flow}
\end{figure}