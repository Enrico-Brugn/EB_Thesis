\chapter{Contributions to the Thesis}

In the following flow charts, the tasks in black are those performed by me and the contribution of other staff and researchers at \acs{ibm} Research Europe - Zurich is highlighted in red. Given the abundance of tasks, the black rectangles represent multiple process steps, further explained in Section~\ref{chap:tase} for the initial nanofabrication process and in the experimental chapters for growth, characterisation, and data analysis. Rhomboids represent single process steps.

\begin{figure}[H]
\centering
\tikzsetnextfilename{nanofab_flow}
\begin{tikzpicture}
    \node[draw,
          blue,
	   rounded rectangle, 
	   minimum width=2.5cm,] (soi){Pristine SOI wafer (commercially available)};
    \node[draw,
          red,
          below = of soi,
          trapezium,
          trapezium left angle = 120,
          trapezium right angle = 120,
          trapezium stretches] (dicing) {Dicing};
    \node[draw,
          below = of dicing,
          trapezium,
          trapezium left angle = 120,
          trapezium right angle = 120,
          trapezium stretches] (mask) {Design of the lithography mask};
    \node[draw,
          red,
          right = of mask,
          trapezium,
          trapezium left angle = 120,
          trapezium right angle = 120,
          trapezium stretches] (optimask) {Fabrication of the optical mask};
    \node[draw,
          below = of mask,
	   minimum width=2.5cm,] (markers){Fabrication of e-beam markers};
    \node[draw,
          red,
          right = of markers,
          trapezium,
          trapezium left angle = 120,
          trapezium right angle = 120,
          trapezium stretches] (exposure1) {e-beam exposure};
    \node[draw,
          below = of markers,
	   minimum width=2.5cm,] (nanostructures){Definition of nanostructures};
    \node[draw,
          red,
          right = of nanostructures,
          trapezium,
          trapezium left angle = 120,
          trapezium right angle = 120,
          trapezium stretches] (exposure2) {e-beam exposure};
    \node[draw,
          below = of nanostructures,
	   minimum width=2.5cm,] (template){Etching and template oxide deposition};
    \node[draw,
          below = of template,
	   minimum width=2.5cm,] (openings){Creation of the openings in the template oxide};
    \node[draw,
          red,
          right = of openings,
          trapezium,
          trapezium left angle = 120,
          trapezium right angle = 120,
          trapezium stretches] (exposure3) {e-beam exposure};
    \node[draw,
          below = of openings,
	   minimum width=2.5cm,] (etch){Etch back of the sacrificial \acs{si} in the template};
    \node[draw,
          below = of etch,
	   minimum width=2.5cm,] (growth){Growth of III-V material inside the templates};
    \node[draw,
          right = of growth,
          rounded rectangle,
	   minimum width=2.5cm,] (end){III-V \acs{tase} nanostructures};

    \draw [-stealth](soi) edge (dicing);
    \draw [-stealth](dicing) edge (mask);
    \draw [-stealth, red] (mask) -- (optimask);
    \draw [-stealth](mask) edge (markers);
    \draw [stealth-, red] (markers) -- (exposure1);
    \draw [-stealth](markers) edge (nanostructures);
    \draw [stealth-, red] (nanostructures) -- (exposure2);
    \draw [-stealth] (nanostructures) edge (template);
    \draw [-stealth] (template) edge (openings);
    \draw [stealth-, red] (openings) -- (exposure3);
    \draw [-stealth] (openings) edge (etch);
    \draw [-stealth] (etch) edge (growth);
    \draw [-stealth] (growth) edge (end);
\end{tikzpicture}
\caption{Nanofabrication flowchart}
\label{fig:nanofab_flow}
\end{figure}

\begin{figure}
    \centering
    \begin{tikzpicture}
        \node[draw,
	       rounded rectangle, 
	       minimum width=2.5cm,] (start){III-V \acs{tase} nanostructures};
        \node[draw,
	       below = of start, 
	       minimum width=2.5cm,] (fib){FIB lamella preparation};
        \node[draw,
	       below = of fib, 
	       minimum width=2.5cm,] (stem){\acs{stem_m} analysis};
    \end{tikzpicture}
    \caption{Caption}
    \label{fig:characterisation_flow}
\end{figure}