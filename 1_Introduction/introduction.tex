\chapter{Introduction}
\label{chap:introduction}

Most of the world's telecommunication infrastructure has abandoned the electron and chosen the photon as its primary information-carrying particle for long-range information exchanges. Indeed, even at medium distances, the first world has seen the same transition occur, with fibre optic cables substituting copper wires up to many city homes' doorsteps \cite{ECBroadband, ETNOBroadband}. In the short distance, most data centres have completed the transition from copper to fibre optics to connect servers \num{10} to \num{15} years ago, and continue to introduce technological developments to improve bandwidth on this optical support \cite{Cheng2018}. 

This trend has continued towards the ultra-short distances of in-board and in-chip interconnects \cite{Benner2005}, where \acl{pic}s (\acs{pic}s) combine optical emitters and absorbers with waveguides directly on the surface of semiconductor chips, slowly but surely encroaching on what has traditionally been the electron's domain \cite{Shekhar2024, Margalit2021, Smit2019}. However, there still exists a point at which the maturity (and overwhelming budget superiority) of electronics requires the photonic data stream to be translated into an electric signal so that transistors can manipulate it in silicon-based \acf{cmos} integrated circuits. 

Research focussing on optically active elements at the interface between light-speed communication and electronic elaboration efficiency has been ongoing for decades, ever since the first concept of \acs{pic} was proposed at the end of the 1960s \cite{Miller1969}, through the development of silicon waveguides in the 1980s \cite{Soref1987}. Given the relatively simple material requirements of electronics compared to photonics, it was not until the rapid diffusion and increasing bandwidth requirements of the Internet became clear and were followed soon after by the limits of Moore's law coming into sight that photonics gained popularity as a complementary technology \cite{Pathak2019}. Since then, photonics has begun to shape the myriad of devices on which we rely so much in modern society and has been proposed as a prime candidate for interconnecting quantum systems \cite{Luo2023}.

\Acl{si} photonics is an emerging technology, but only in the sense that it has recently started to find its place in the manufacturing sector of the semiconductor industry. Since all of the materials involved are already being used in the \acs{cmos} process, the integration of new components into the \acf{pdk} of many foundries was straightforward \cite{Shi2022, Siew2021, Novack2014}. However, the advantages of \acl{si} photonics are much more related to the high maturity of \acl{si} processing technology \cite{Novack2014}, which has been the backbone of the computing infrastructure of the world since the invention of the \acs{cmos} process, than with the actual material properties of this semiconductor. \Acl{si}'s indirect bandgap makes it an ill-suited material for light emission and adsorption, at least when it is not alloyed or otherwise modified.

On the other hand, III-V semiconductors have been the backbone material of solid-state light emission \cite{Schlereth1996, Nakamura1994} and sensing \cite{Ting2019, Liang2022} devices for the past three decades. Group III and V elements are rare and often toxic compared to the group IV element \acs{si}, which is one of the most readily available on the planet. However, in what is a key difference, \acs{si} has an indirect band gap \cite{Chelikowsky1974}, which enables the light absorbing and emitting properties of III-V semiconductors, which have a direct band gap \cite{Khanin2005}, to shine in comparison.

Today, industrial integration of III-V components on a \acl{si} platform is mainly achieved through the transfer of active components from the lattice-matched substrate on which they grew \cite{Han2022}. Research and development of this type of integration is ongoing and focussing more and more on pick-and-place methods \cite{McPhillimy2020} to maximise positioning accuracy, which is a key metric to reduce the footprint and energy efficiency of optical interconnects compared to their traditional metallic counterparts \cite{Miller2000}. In fact, interconnect energy costs are rapidly approaching parity with computational energy costs in modern electronics, adding to the environmental burden of our civilisation \cite{Miller2009}. Therefore, the introduction of a new technology must demonstrate clear advantages in this field.

Although the above-described heterogeneous integration solution sidesteps the lattice mismatch issue between III-V semiconductors and \acl{si}, it incurs the positioning problem. The second approach is to focus on monolithic integration, which involves the direct growth of III-V material on a silicon-based substrate. This approach reduces the positioning fluctuation to a few nanometres but leaves the scientist with a much larger problem to solve: that of lattice-mismatch-borne defects. Technological advancements such as \acf{sag} and \acf{art} \cite{Kunert2016, Fiorenza2019}, together with the selection of the appropriate starting facet \cite{Kunert2018} have sought to minimise the occurrence of such defects. A very powerful evolution of all of these methods is \acf{tase}, a \acf{soi} and \acf{mocvd} based and \acs{cmos}-compatible monolithic integration method \cite{borgTASEp2018, Schmid2015}.

Another great advantage of \acs{tase} is that it enables defect control in the resulting material \cite{Knoedler2017, Staudinger2018, Han2020}. During the study of this growth method, two key findings on which the present work is built were observed for the III-V growth in templates. The first is the high influence of the V/III ratio loaded in the reactor on the morphology of the resulting nanowire because of the dependence of facet-specific growth rates on this metric. High V/III ratios in the growth of \acs{inas} were shown to favour the stabilisation of a \hkl{1 1 1} B facet, while low V/III ratios favour the stabilisation of \hkl{1 1 0} facets, with intermediate values creating a multi-faceted configuration \cite{Borg2015}. The second showed how different facets enable different rates of III element incorporation in ternary materials such as \acs{ingaas}, leading to internal composition gradients in \acs{tase} nanowires and microstructures \cite{Borg2019}.

Despite these two findings, the majority of studies related to \acs{tase} nanowires show structures containing multi-faceted growth fronts \cite{Knoedler2017, Han2016, Yan2021, Scherrer2024, Xue2021} and the studies that showed facet stabilisation in templates dealt primarily with homoepitaxy \cite{Goswami2020, Brunelli2019}. This can be motivated by the desire to mimic the homoepitaxial result showing the stabilisation of \hkl{1 1 0} facets, as this has been shown to lead to defect-free crystals. However, due to the nature of the \acs{tase} fabrication process leading to \acs{mocvd} deposition, the selection of one or more \hkl{1 1 0} facets has proven to be a stochastic process \cite{Knoedler2017} and, over time, this research path was left to rest, at least at \acs{ibm} Research Europe - Zurich.

This also resulted in little attention being paid to the second alternative, the study of growth with a single \hkl{1 1 1}-stabilised facet. In this thesis, I will attempt to highlight the advantages of this method for the fabrication of binary phosphide / ternary arsenide heterostructures in \acs{tase} nanowires, estimate their growth yield, and study how this method applies to more complex structures.

\paragraph{Structure of the thesis} After this general introduction contextualising the research carried out in this project within the knowledge base and the funding objectives, Chapter~\ref{chap:review} aims to dive deeper into the state-of-the-art and introduce concepts necessary to understand the techniques used in the fabrication and characterisation of III-V devices and their material properties. Chapter~\ref{chap:growth} talks about the initial experiments and how they led to the formulation of the facet stabilisation method used throughout the thesis, while Chapter~\ref{chap:properties} explores the consequences of this growth regimen in the growth of III-V micro- and nanostructures in \acs{tase} templates. Finally, Chapter~\ref{chap:yield_analysis} aims to explore the growth yields that can be obtained with the facet stabilisation method and proposes an avenue for automatic yield assessment for future samples. The thesis concludes in Chapter~\ref{chap:conclusions} with an overview of the results and proposals for further avenues of study. 

Appendix~\ref{chap:tase} summarises the \acs{tase} process as used for the fabrication of the samples shown in this thesis, while Appendix~\ref{chap:tools} gives an overview of the tools and facilities employed.

\section{\texorpdfstring{The \acs{design} consortium}{The DESIGN-EID consortium}}

The work described in this thesis was carried out within the framework of the "\acl{design}" (\acs{design}) project. The project was made possible thanks to funding from the \acl{eu}'s HORIZON2020 programme through grant number 860095 \cite{CordisDESIGN}.
\par
\Acs{design} brings together three institutions: the University of Glasgow, located in Glasgow, Scotland, as the leading institution and primary beneficiary of the \acs{eu} funding; \acs{ibm} Research Europe - Zurich, located in Rueschlikon, Switzerland, as the leading experimental partner; and Synopsys Quantum ATK, located in Copenhagen, Denmark, as the leading simulation partner.
\par
Aside from its training objectives, the project aimed to explore, characterise, simulate, and exploit defects in III-V semiconductors and the effect of growth conditions in their formation. To achieve these goals, four \acl{wp}s (\acs{wp}) were identified:
\begin{itemize}
    \item \acs{wp}1: Material Growth. This package encompassed crystal growth and characterisation on the nano- and microscale. In particular, crystalline and compositional studies were to be performed.
    \item \acs{wp}2: Material Simulation. This package dealt with the simulation of the effect of crystalline defects and compositional gradients on the electronic and optical properties of the material. Another topic of interest was the simulation of the growth dynamics leading to defect formation.
    \item \acs{wp}3: Device Fabrication. As the name suggests, the objective of this package was to employ the knowledge gained during the project to fabricate a device that could be used as an active photonic component.
    \item \acs{wp}4: Device Simulation. This package encompasses the simulation effort to identify the best device structure and composition to achieve its intended role as an active photonic component.
\end{itemize}
\par
Three \acl{esr}s (\acs{esr}s) were hired to achieve the consortium's objective, and each of them carried out a third of their doctoral studies at the University of Glasgow and spent the remaining time at one of the other two partner institutions. As \acs{esr}1, I spent the first two years of my doctoral studies at \acs{ibm} Research Europe- Zurich, from September 2020 to the end of August 2022, mainly performing experimental work related to \acs{wp}1. Finally, in the last year of my doctorate, I concentrated on data analysis and developing a machine-learning algorithm for classifying \acs{tase}-grown III-V nanowires from \acf{sem_m} images.