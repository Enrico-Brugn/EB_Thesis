\chapter{Introduction}
\label{chap:introduction}

Most of the world's telecommunication infrastructure has abandoned the electron and chosen the photon as its primary information-carrying particle for long-range information exchanges. Indeed, even at medium distances, the first world has seen the same transition occur, with fibre optic cables substituting copper wires up to many city homes' doorsteps \cite{ECBroadband, ETNOBroadband}. In the short distance, most data centres have completed the transition from copper to fibre optics to connect servers with each other \num{10} to \num{15} years ago, and continue to introduce technological developments to improve bandwidth on this optical support \cite{Cheng2018}. 

Further towards the ultra-short distances of in-board and in-chip interconnects, \acl{pic}s (\acs{pic}s) combine optical emitters and absorbers with waveguides directly on the surface of semiconductor chips, slowly but surely encroaching on what has traditionally been the electron's domain \cite{Shekhar2024, Margalit2021, Smit2019}. However, there still exists a point at which the maturity (and overwhelming budget superiority) of electronics requires the photonic data stream to be translated into an electric signal, so that transistors can manipulate it in silicon-based \acf{cmos} circuits. 

Research focussing on the optically active elements at the interface between light-speed communication and electronic elaboration efficiency has been ongoing for decades and is constantly shaping the myriad of devices upon which we have come to rely on so much in modern society.

\begin{itemize}
    \item General Si photonics and what is used now
    \item advantages of moving to III-V
    \item challenges facing III-V monolithic integration
    \item focus on heterostructure control inside TASE
    \item chapter descriptions
\end{itemize}

\section{\texorpdfstring{The \acs{design} consortium}{The DESIGN-EID consortium}}

The work described in this thesis was carried out within the framework of the "\acl{design}" (\acs{design}) project. The project was made possible thanks to funding from the \acl{eu}'s HORIZON2020 programme through grant number 860095 \cite{CordisDESIGN}.
\par
\Acs{design} brings together three institutions: the University of Glasgow, located in Glasgow, Scotland, as the leading institution and primary beneficiary of the \acs{eu} funding; \acs{ibm} Research Europe - Zurich, located in Rueschlikon, Switzerland, as the leading experimental partner; and Synopsys Quantum ATK, located in Copenhagen, Denmark, as the leading simulation partner.
\par
Aside from its training objectives, the project aimed to explore, characterise, simulate, and exploit defects in III-V semiconductors and the effect of growth conditions in their formation. To achieve these goals, four \acl{wp}s (\acs{wp}) were identified:
\begin{itemize}
    \item \acs{wp}1: Material Growth. This package encompassed crystal growth and characterisation on the nano- and microscale. In particular, crystalline and compositional studies were to be performed.
    \item \acs{wp}2: Material Simulation. This package dealt with the simulation of the effect of crystalline defects and compositional gradients on the electronic and optical properties of the material. Another topic of interest was the simulation of the growth dynamics leading to defect formation.
    \item \acs{wp}3: Device Fabrication. As the name suggests, the objective of this package was to employ the knowledge gained during the project to fabricate a device that could be used as an active photonic component.
    \item \acs{wp}4: Device Simulation. This package encompasses the simulation effort to identify the best device structure and composition to achieve its intended role as an active photonic component.
\end{itemize}
\par
Three \acl{esr}s (\acs{esr}s) were hired to achieve the consortium's objective, and each of them carried out a third of their doctoral studies at the University of Glasgow and spent the remaining time at one of the other two partner institutions. As \acs{esr}1, I spent the first two years of my doctoral studies at \acs{ibm} Research Europe- Zurich, from September 2020 to the end of August 2022, mainly performing experimental work related to \acs{wp}1. Finally, in the last year of my doctorate, I concentrated on data analysis and developing a machine-learning algorithm for classifying \acs{tase}-grown III-V nanowires from \acf{sem_m} images.