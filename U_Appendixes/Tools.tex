\chapter{Fabrication and Characterization Tools}
\label{chap:tools}

The \acf{brnc} is a facility at \acs{ibm} Research Europe – Zurich, in Rueschlikon, Zurich, Switzerland. It consists of a cleanroom and Noise-free lab. All the experimental work on the growth and characterisation of nanostructures presented in this work took place at \acs{ibm} Research Europe - Zurich, most of it within the \acs{brnc} itself. 

\section{Cleanroom Tools}

The \acs{brnc} cleanroom is a \qty{950}{\m^2} area with different rooms ranging in cleanroom classification from 100 (ISO 5) to 10’000 (ISO 7). Most of the chemicals and tools used during the \acf{tase} process and sample preparation are within these rooms. 
\par
The following tools, located in the \acs{brnc} cleanroom, were used:
\begin{itemize}
    \item wet benches, as fume hoods are impractical in a cleanroom: while in fume hoods, the vapours and gasses are aspirated upwards, on the wet benches they are evacuated through the work surface, thanks to holes in it and an applied negative pressure gradient under it. These workstations have smaller tools, such as spin coaters for resist application, hot plates, and ultrasound baths. Solution-based etching can also be carried out on wet benches safely with different dangerous chemicals such as strong acids and bases and toxic reagents,
    \item \acf{pecvd} tool: Oxford Plasma Pro 100 for \acl{sio2} deposition,
    \item \acf{ald} tool: Oxford FlexAL,
    \item \acf{rie} tool: Oxford Plasma Pro NPG 80, which employs a chemically active plasma to etch oxides or metal,
    \item \acf{icp} etching tool: Oxford Plasmalab System 100. This tool employs a chemically active inductively coupled plasma to achieve extremely anisotropic etching of \acl{si},
    \item mask aligner: Seuss Mask Aligner MA6 to execute photolithography steps with pre-written metal masks,
    \item rapid thermal annealer Annealsys AS-one 150 for high-temperature annealing treatments of deposited material layers,
    \item dual beam \acf{fib} tool: FEI Helios NanoLab 450S, for SEM imaging; and lamella cutting, manipulation, and thinning,
    \item plasma cleaner PVA TePla Plasma Asher for the removal of resist traces and surface preparation,
    \item reflectometer: Nanospec II 150-VIS to measure the thickness of deposited material layers,
    \item multiple optical microscopes to monitor the fabrication process,
    \item Raman spectroscope: ND-MDT NTEGRA Spectra.
\end{itemize}
 
\section{Noise-Free Labs Tools}

The \acs{brnc} Noise-free labs consist of \qty{176}{\m^2} of noise-shielded laboratory area. The shielding limits vibration noise to \qty{300}{\nm\per\s} at \qty{1}{Hz}, and to \qty{10}{\nm\per\s} at frequencies higher than \qty{100}{Hz}; temperature fluctuations to \qty{0.01}{\degreeCelsius}; humidity variations to \qty{2}{\%}; acoustic noise level to \qty{21}{dBc}; and electromagnetic noise to \qty{0.3}{\nano\tesla} AC, and to \qty{15}{\nano\tesla} DC \cite{Lörtscher_Noise_Free}. These technical specifications are instrumental in successfully recording atomic-resolution \acf{stem_m} images, making the laboratory rooms a tool in their own right.
\par
Two of the tools in the Noise-free labs were used in this work, relating to the \acs{stem_m} measurements I carried out. The first is the plasma cleaner Fischione Model 1020 for the final oxygen-argon plasma cleaning (effective in removing hydrocarbon contamination) of the lamella before insertion in the \acs{stem_i}. 
\par 
The \acl{stem_i} JEOL ARM200F is the second tool. This microscope is equipped with a cold field emission gun placed in a \qty{200}{\kV} extraction and acceleration potential drop (emitting with an energy width of less than \qty{0,3}{eV}), hexapole optics, \acf{bf}, \acf{adf}, and \acf{haadf} detectors. Two spectroscopic techniques are also available on this microscope in the form of \acf{eds} and \acf{eels}. The \acs{eds} tool JED-2300T was used to characterise the samples. This tool is equipped with a \qty{100}{\mm^2} detector surface that allows for very fast \acs{eds} mapping. The detector is equipped with an ultra-thin window, which provides excellent performance for analysing light elements. Resolution for the Mn K$\alpha$ line is less than \qty{138}{\eV}. \qty{1}{\steradian} collection solid angle. This tool can also be operated as a \acf{tem_i}.

\section{Other Tools}

Outside both the cleanroom and Noise-free lab, the following tools were instrumental in achieving the experimental results:
\begin{itemize}
    \item \acf{sem_i}: Hitachi SU8000 type II to monitor the fabrication process,
    \item a “home-built” \acf{mocvd} setup in which the III-V nanowires were grown. The vertical reactor contains a showerhead, which introduces precursors in the chamber. The rotating susceptor can accommodate three 2-inch wafers, or smaller pieces, such as the \qtyproduct{2 x 2}{\cm} dices used in this work, utilising carriers \cite{Brugnolotto2023}.
\end{itemize}
\par
The precursors used in the growth of III-V material are:
\begin{itemize}
    \item \acf{tmin},
    \item \acf{tmga},
    \item \acf{tbas},
    \item \acf{tbp},
    \item \acf{tmsb}
\end{itemize}

The \acf{gms} software was used to record and later analyse the microscopy images and \acs{eds} data. L-Edit and KLayout were used to design and visualise the \acf{ebl} and optical lithography masks in GDSII format. 
\par 
Tools that I did not operate personally but were used in the fabrication or characterisation of my samples were:
\begin{itemize}
    \item sputter tool: Von Ardenne CS 320S to deposit the tungsten that makes up the \acl{ebl} markers,
    \item \acl{ebl} tool Vistec EBPG 5200+ for exposure of electron-sensitive resists,
    \item laser writer: Heidelberg DWL 2000 for the creation of optical lithography masks (glass substrate, metallic layer mask),
    \item dicer tool: ADT ProVectus LA 7100 for dicing large wafers into smaller dices,
    \item a “home-built” photoluminescence spectroscope for photoluminescence characterisation.
\end{itemize}